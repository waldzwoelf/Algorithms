\chapter{Proving the Correctness of an Algorithm}
In this chapter we will show two different methods that can be used to prove that an algorithm is correct.
\begin{enumerate}[(a)]
\item The method of \blue{computational induction} can be used to verify the correctness of an algorithm that
      is defined recursively.
\item In order to establish the correctness of an algorithm that is defined iteratively we use \blue{symbolic
      execution}. 
\end{enumerate}

\section{Computational Induction}
Figure \ref{fig:power.py} shows the definition of the function $\mytt{power}(m,n)$ that computes
the value $m^n$.  We have already analyzed the computational complexity of this program in the last chapter.
In this chapter we show how the correctness of this function can be verified.

\begin{figure}[!h]
  \centering
\begin{minted}[ frame         = lines, 
                framesep      = 0.3cm, 
                numbers       = left,
                numbersep     = -0.2cm,
                bgcolor       = sepia,
                xleftmargin   = 0.8cm,
                xrightmargin  = 0.8cm
              ]{python3}
    def power(m, n):
        if n == 0:
            return 1
        p = power(m, n // 2)
        if n % 2 == 0:
            return p * p
        else:
            return p * p * m
\end{minted}
\vspace*{-0.3cm}
  \caption{Computation of $m^n$ for $m,n \in \N$.}
  \label{fig:power.py}
\end{figure} 

It is by no means obvious that the program shown in \ref{fig:power.stlx} does compute
$m^n$.  We prove this claim by  \blue{computational induction}\index{computational induction}.
Computational induction is an induction on the number of recursive invocations.
This method is the method of choice to prove the correctness of a function that is defined recursively.
The method of computational induction consists of three steps:
\begin{enumerate}
\item The \blue{base case}.

      In the base case we have to show that the function definition is correct in all those cases where the function
      does not invoke itself recursively.
\item The \blue{induction step}.

      In the induction step we have to prove that the function definition works in all those cases where
      the function does invoke itself recursively.  In order to prove the correctness of these cases we may
      assume that the recursive invocations work correctly.  This assumption is called the
      \blue{induction hypotheses}.
\item The \blue{termination proof}.

      In this final step we have to prove that the recursive definition of the function is \blue{well founded},
      i.e.~we have to prove that the recursive invocations terminate.
\end{enumerate}
Let us prove the claim 
\\[0.2cm]
\hspace*{1.3cm}
 $\mytt{power}(m,n) = m^n$
\\[0.2cm] 
by computational induction.
\begin{enumerate}
\item \textbf{Base case}:

      The only case where \mytt{power} does not invoke itself recursively is the case $n = 0$.  
      In this case, we have
      \\[0.2cm]
      \hspace*{1.3cm} 
      $\mytt{power}(m,0) = 1 =  m^0$. \mycheck
\item \textbf{Induction step}:

      The recursive invocation of $\mytt{power}$ has the form
      $\mytt{power}(m,n \dv 2)$.  By the induction hypotheses we know that 
      \\[0.2cm]
      \hspace*{1.3cm}
      $\displaystyle \mytt{power}(m,n \dv 2) = m^{n \dvv 2}$ 
      \\[0.2cm]
      holds.  After the recursive invocation there are two different cases:
      \begin{enumerate}
      \item $n \;\mytt{\%}\; 2 = 0$, therefore $n$ is even.

            Then there exists a number $k \in \N$ such that $n = 2 \cdot k$ and therefore
            $n \dv 2 = k$.
            Hence we have:
            \\[0.2cm]
            \hspace*{1.3cm}
           $ 
            \begin{array}{lcl}
            \mytt{power}(m,n) & = & \mytt{power}(m,k) \cdot \mytt{power}(m,k) \\[0.2cm]
                                & \stackrel{\mathrm{IV}}{=} & m^k \cdot m^k  \\[0.2cm]
                                & = & m^{2\cdot k} \\[0.2cm]
                                & = & m^{n}.
            \end{array}
            $            
      \item $n \;\mytt{\%}\; 2 = 1$, therefore $n$ is odd.

            Then there exists a number $k \in \N$ such that $n = 2 \cdot k + 1$ and we have
            $n \dv 2 = k$.  In this case we have:
            \\[0.2cm]
            \hspace*{1.3cm}
            $ 
            \begin{array}{lcl}
            \mytt{power}(m,n) & = & \mytt{power}(m,k) \cdot \mytt{power}(m,k) \cdot m  \\[0.2cm]
                                & \stackrel{\mathrm{IV}}{=} & m^k \cdot m^k \cdot m  \\[0.2cm]
                                & = & m^{2\cdot k+1} \\[0.2cm]
                                & = & m^{n}.
            \end{array}
            $
      \end{enumerate}
      As we have shown that $\mytt{power}(m,n) = m^n$ in both cases, the induction step is finished. \mycheck
\item \textbf{Termination proof}:
      Every time the function \mytt{power} is invoked as $\mytt{power}(m, n)$ and $n > 0$, the recursive
      invocation has the form $\mytt{power}(m,n \dv 2)$ and, since $n \dv 2 < n$ for all $n > 0$, the second
      argument is decreased.  As this argument is a natural number, it must eventually reach $0$.  But if the
      second argument of the function $\mytt{power}$ is $0$, the function terminates. \mycheck
      \qed
\end{enumerate}

\begin{figure}[!h]
  \centering
\begin{minted}[ frame         = lines, 
                framesep      = 0.3cm, 
                numbers       = left,
                numbersep     = -0.2cm,
                bgcolor       = sepia,
                xleftmargin   = 0.8cm,
                xrightmargin  = 0.8cm
                ]{python3}
    def div_mod(m, n):
        if m < n:
            return 0, m
        q, r = div_mod(m // 2, n)
        if 2 * r + m % 2 < n:
            return 2 * q, 2 * r + m % 2
        else:
            return 2 * q + 1, 2 * r + m % 2 - n                
\end{minted}
\vspace*{-0.3cm}
  \caption{The function \mytt{div\_mod}.}
  \label{fig:div_mod}
\end{figure} 

\exercise
Prove that the function \mytt{div\_mod} that is shown in Figure \ref{fig:div_mod} satisfies the specification
\\[0.2cm]
\hspace*{1.3cm}
$\mytt{div\_mod}(m, n) = \langle q, r \rangle \rightarrow m = q \cdot n + r \wedge r < n$. \eoxs


\section{Symbolic Execution}
In the last chapter we have seen how to prove the correctness of a recursive function via
\blue{computational induction}.  If a function is implemented via loops instead of recursion, then the method
of computational induction is not applicable.  Therefore, this section introduces the method of
\blue{symbolic execution}. \index{symbolic execution}  Using this method it is possible to verify the
correctness of programs that are implemented in an iterative fashion using loops. 
We will introduce this method via a simple example.  Consider the program shown in Figure
\ref{fig:power-iterative-annotated.stlx}. 

\begin{figure}[!h]
\centering
\begin{Verbatim}[ frame         = lines, 
                  framesep      = 0.3cm, 
                  labelposition = bottomline,
                  numbers       = left,
                  numbersep     = -0.2cm,
                  xleftmargin   = 1.3cm,
                  xrightmargin  = 1.3cm,
                  codes         = {\catcode`_=8\catcode`$=3},
                  commandchars  = \\\{\},
                ]
    def power(x$_1$, y$_1$):
        r$_1$ = 1
        while y$_n$ > 0:
            if y$_n$ % 2 == 1:
                r$_{n+1}$ = r$_n$ * x$_n$
            x$_{n+1}$ = x$_n$ * x$_n$
            y$_{n+1}$ = y$_n$ \symbol{92} 2
        return r$_N$
\end{Verbatim}
\vspace*{-0.3cm}
\caption{An annotated programm to compute powers.}
\label{fig:power-iterative-annotated.stlx}
\end{figure} % $

The main difference between a mathematical formula and a program is that in a formula all
occurrences of a variable refer to the same value.   This is different in a program because the
variables change their values dynamically.  In order to deal with this property of program variables
we have to be able to distinguish the different occurrences of a variable.  To this end,  we 
\blue{index} the program variables. 
When doing this we have to be aware of the fact that the same occurrence of a program variable can
still denote different values if the variable occurs inside  a loop.  In this case we have to index
the variables in a way that the index includes a counter that counts the number of loop iterations.
For concreteness, consider the  program shown in 
Figure \ref{fig:power-iterative-annotated.stlx}.  
Here, in line 5 the variable \mytt{r} has the index $n$ on the right side of the assignment,
while it has the index $r_{n+1}$ on the left side of the assignment in line 5.  The index $n$ denotes 
the number of times that the test \mytt{y$_n$ > 0} of the \mytt{while} loop has been executed.
After the \texttt{while}-loop finishes, the variable $\mytt{r}$ is indexed as
$\mytt{r}_N$ in line 8, where $N$ denotes the total number of times that the test \mytt{y > 0} has been executed.
We show the correctness of the given program next.  Let us define
\\[0.2cm]
\hspace*{1.3cm}
$ a := x_1, \quad b := y_1$.
\\[0.2cm]
We will show that the \mytt{while} loop satisfies the invariant
\begin{equation}
  \label{eq:powerInv}
   r_n \cdot x_n^{y_n} = a^b.
\end{equation}
This claim is proven by induction on the number of loop iterations.
\begin{enumerate}
\item[B.C.:] $n=1$.

            Since we have $r_1 = 1$, $x_1 = a$, and $y_1 = b$ we have 
            \\[0.2cm]
            \hspace*{1.3cm}
            $r_n \cdot x_n^{y_n} = r_1 \cdot x_1^{y_1} = 1 \cdot a^{b} = a^b$.
\item[I.S.:] $n \mapsto n + 1$.

            We proof proceeds by a case distinction with respect to the expression $y \mmod 2$:
            \begin{enumerate}
            \item $y_n \mmod 2 = 1$.

                  Then we have $y_{n} = 2 \cdot (y_n\dv 2) + 1$ and
                  $r_{n+1} = r_n \cdot x_n$.  Hence
                  \begin{eqnarray*}
                      &   & r_{n+1} \cdot x_{n+1}^{y_{n+1}} \\[0.2cm] 
                      & = & (r_{n} \cdot x_n) \cdot (x_{n} \cdot x_{n})^{y_{n}\dvv 2} \\[0.2cm] 
                      & = & r_{n} \cdot x_n^{2 \cdot (y_{n}\dvv 2) + 1} \\[0.2cm] 
                      & = & r_{n} \cdot x_n^{y_n} \\
                      & \stackrel{i.h.}{=} & a^{b} 
                  \end{eqnarray*}

            \item $y_n \mmod 2 = 0$.

                  Then we have $y_{n} = 2 \cdot (y_n\dv 2)$ and $r_{n+1} = r_n$.
                  Therefore
                  \begin{eqnarray*}
                      &   & r_{n+1} \cdot x_{n+1}^{y_{n+1}} \\[0.2cm] 
                      & = & r_{n} \cdot (x_{n} \cdot x_{n})^{y_{n}\dvv 2} \\[0.2cm] 
                      & = & r_{n} \cdot x_n^{2 \cdot (y_{n} \dvv 2)} \\[0.2cm] 
                      & = & r_{n} \cdot x_n^{y_n} \\
                      & \stackrel{i.h.}{=} & a^{b} 
                  \end{eqnarray*}
            \end{enumerate}
\end{enumerate}
This shows the validity of equation (\ref{eq:powerInv}).   If the \mytt{while} loop
terminates, we must have $y_N = 0$.  If $n=N$, then equation (\ref{eq:powerInv}) yields:
\\[0.2cm]
\hspace*{1.3cm}
$$
\begin{array}[t]{cl}
                 & r_N \cdot x_N^{y_N} = a^b \\
\Leftrightarrow  & r_N \cdot x_N^{0}  = a^b \\
\Leftrightarrow  & r_N \cdot 1       = a^b \\
\Leftrightarrow  & r_N               = a^b
\end{array}
$$
\\[0.2cm]
This shows $r_N = a^b$ and since it is obvious that the \mytt{while} loop terminates, we have proven that
$\mytt{power}(a,b) =a^b$ holds. \qed

\exercise
Use the method of symbolic program execution to prove the correctness of the implementation of the
Euclidean algorithm that is shown in Figure \ref{fig:gcd.stlx} on page \pageref{fig:gcd.stlx}.  During the proof
you should make use of the fact that for all positive natural numbers $x$ and $y$ the equation
\\[0.2cm]
\hspace*{1.3cm}
$\mytt{gcd}(x, y) = \mytt{gcd}(x \,\mytt{\%}\, y, y)$
\\[0.2cm]
is valid.  

\begin{figure}[!ht]
\centering
\begin{Verbatim}[ frame         = lines, 
                  framesep      = 0.3cm, 
                  firstnumber   = 1,
                  labelposition = bottomline,
                  numbers       = left,
                  numbersep     = -0.2cm,
                  xleftmargin   = 0.8cm,
                  xrightmargin  = 0.8cm,
                ]
    def gcd(x, y):
        while y != 0:
            x, y = y, x % y
        return x
\end{Verbatim}
\vspace*{-0.3cm}
\caption{The Euclidean algorithm.}
\label{fig:gcd.stlx}
\end{figure}
\pagebreak

\section{Check Your Understanding}
If you were able solve the  exercises given in this chapter, then you should have mastered the concepts
that have been introduced in this chapter.


%%% Local Variables:
%%% mode: latex
%%% TeX-master: "algorithms"
%%% End:
